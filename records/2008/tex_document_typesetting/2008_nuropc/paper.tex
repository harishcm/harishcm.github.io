\documentclass{nuropc}
\usepackage{color}
\usepackage{graphicx}
\usepackage{natbib}
\usepackage{url}
\usepackage[version=3]{mhchem}
\usepackage{booktabs}
\usepackage{dcolumn}
 \newcolumntype{.}{D{.}{.}{2.2}}

\setcounter{secnumdepth}{0}

\title{Basic behaviour study of cement treated Singapore marine clay}

\author{Mohanadas H. C. \\
\normalsize\textit{Department of Civil Engineering, 
National University of Singapore \\ 
Block E1A \#07-03, 1 Engineering Drive 2, Singapore 117576}}

\begin{document}
\maketitle
\begin{abstract}
A study was conducted to investigate 
the effects of curing time on the 
basic behaviour of cement treated Singapore marine clay. 
Specimens were prepared in 
two different soil-cement-total water content ratios and 
left to cure under water for periods of 7~days, 4~weeks and 3~months. 
After curing, they were subject to a series of 
triaxial tests---unconfined compressive strength tests~(UCT), 
isotropic compression tests~(ICT), isotropically consolidated 
undrained tests~(CIU) and isotropically consolidated drained~(CID) tests. 
The effects of curing time on 
stress-strain and strength behaviour was investigated. 
\end{abstract}

\section{Introduction}
Singapore Upper Marine Clay located 4--5\,m beneath the seabed, 
retrieved from a dredge site in Pulau Tekong, Singapore 
was used in this study. 
The specimens were mixed in soil-cement-total water content 
ratios of 2-1-4 and 5-1-6.
To prepare the clay-cement mixture, the required amount of cement slurry 
was added to 100\% water content remoulded clay and 
mixed in a Hobart mixer for about 10\,min. 
The clay-cement mixture was then cured under water in PVC split 
molds of diameter 50\,mm and length 100\,mm. 
The specimens were trimmed down to
diameter 38\,mm by length 76\,mm before testing. 
Parameters for the triaxial 
tests are given in Table~\ref{parameters}. 
The UCT, CIU and CID tests followed 
standards for soil testing from the 
British Standards Institute \citep{bs1377}. 

\begin{table}[htbp]
\caption{Parameters for triaxial tests.}
\begin{center}
\begin{tabular}{l....} \toprule
 & \multicolumn{1}{c}{UCT} & \multicolumn{1}{c}{ICT} & \multicolumn{1}{c}{CIU}
& \multicolumn{1}{c}{CID} \\ \midrule
Side drains &  & \multicolumn{1}{c}{Present} & \multicolumn{1}{c}{Present} 
& \multicolumn{1}{c}{Present} \\ 
Membrane thickness (mm)          &   & 0.2 & 0.2  & 0.2 \\ 
Axial displacement rate (mm/min) & 1 &     & 0.01 & 0.007 \\ 
Isotropic loading rate (kPa/min) &   & 1   &      & \\ \bottomrule
\end{tabular}
\end{center}
\label{parameters}
\end{table}

\section{Literature survey}
The soil-cement reaction mechanism involves two distinct reactions 
resulting in primary and secondary cementitious products. 
There is a rapid hydration reaction and a much slower pozzolanic reaction. 
The rapid hydration reaction occurs from the moment water is added to cement. 
It results in the primary cementitious products of 
hydrated calcium silicates ($\ce{C2SH_{x}}$, $\ce{C3S2H_{x}}$), 
hydrated calcium aluminates ($\ce{C3AH_{x}}$, $\ce{C4AH_{x}}$) and 
hydrated lime $\ce{Ca(OH)2}$. 
The hydrated lime then dissociates and 
raises the pH value of the pore water. 
This causes the pozzolanic reaction in which 
silica and alumina in the soil dissolve and 
react with the now free $\ce{Ca^2+}$ ions. 
The pozzolanic reaction results in the secondary cementitious products of 
hydrated calcium silicates and hydrated calcium aluminates. 

\citet{strength-compression} found that an increase in the curing time of 
cement treated Singapore marine clay 
from 7~days to 4~weeks resulted in a higher unconfined compressive strength. 
Findings from (uniaxial) oedometer consolidation tests on Singapore marine 
clay revealed that the change in the compression index, 
$\mathrm{C_c}$, and recompression index, $\mathrm{C_r}$, 
between 7~days and 4~weeks was "rather insignificant" \citep{compression}.
The current study explores the curing time effect on specimens 
under triaxial conditions with curing times of up to 3~months.

Cement treatment can be done in the practice through 
deep mixing and jet grouting. 
Both the mixing ratios used in this study 
were within the range of that used in some previous studies
on deep mixing and jet grouting \citep{cement-contents}.

\begin{figure}[htbp]
\centering
\input{./uct-multi.tex}
\caption{Results of UCT tests for (a) 2-1-4 specimens and (b) 5-1-6 specimens.}
\label{fig:UCT}
\end{figure}

\section{Unconfined compressive strength tests}
Results from the UCT tests are given in Fig~\ref{fig:UCT}.
There is a significant increase in the unconfined compressive strength 
from 7~days to 4~weeks and a further increase from 4~weeks to 3~months.
The average rate of this increase between 4~weeks and 3~months was 
lower than $frac{1}{3}$ of the average rate between 7~days and 4~weeks.
This was consistent for specimens of both mixing ratios. 
Thus, there seems to be a modulation this curing time effect 
after 4~weeks of curing. 

\begin{figure}[htbp]
\begin{center}
\input{./ict-214.tex}
\end{center}
\begin{center}
\input{./ict-516.tex}
\end{center}
\caption{Compression space curves from ICT tests on (a) 2-1-4 and (b) 5-1-6 
         specimens.}
\label{fig:ICT}
\end{figure}

\section{Isotropic compression tests}


\begin{figure}[htbp]
\begin{center}
\input{./ciu-214.tex}
\end{center}
\begin{center}
\input{./ciu-516.tex}
\end{center}
\caption{Stress-strain curves from CIU tests on (a) 2-1-4 and (b) 5-1-6 
         specimens. Confining pressure is given next to curing time.}
\label{fig:CIU}
\end{figure}

\section{Consolidated-undrained triaxial compression tests}
Figure \ref{fig:CIU} show an increase in 
the peak strength and the stiffness of specimens
with an increase in the curing time.
The average rate of both these curing time effects 
is are lower between 4~weeks and 3~months 
as opposed to 7~days and 4~weeks.
Thus, there seems to be a modulation in
both these curing time effects after 4~weeks of curing.

\begin{figure}[htbp]
\begin{center}
\input{./cid-vol-214.tex}
\end{center}
\begin{center}
\input{./cid-vol-516.tex}
\end{center}
\caption{Compression space curves from CIU tests on (a) 2-1-4 and (b) 5-1-6 
         specimens. Confining pressure is given next to curing time.}
\label{fig:CID-vol}
\end{figure}

\begin{figure}[htbp]
\begin{center}
\input{./cid-dev-214.tex}
\end{center}
\begin{center}
\input{./cid-dev-516.tex}
\end{center}
\caption{Deviator stress-strain curves from CID tests on 
         (a) 2-1-4 and (b) 5-1-6 specimens. 
         Confining pressure is given next to curing time.}
\label{fig:CID-dev}
\end{figure}

\section{Consolidated-drained traiaxial compression tests}
Figure \ref{fig:CID-vol} shows the compression space curves 
from CID tests. 
There is a levelling in the slope of the curves 
for specimens with longer curing times.
This corresponds to a decrease in the 
isotropic compression index at longer curing times.

\bibliographystyle{apalike}
\bibliography{urop-report}
\end{document}
